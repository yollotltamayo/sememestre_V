\documentclass{article}
\usepackage[utf8]{inputenc}
\usepackage{fancyhdr}
\usepackage{mathtools}
\usepackage{unicode-math}
\usepackage{hyperref}
\usepackage[
top  = 3.15cm,
bottom = 3.15cm,
left   = 3.15cm,
right  = 3.15cm]{geometry}

\pagestyle{fancy}
\fancyhf{}
\setlength{\headheight}{35pt}

\begin{document}
\rhead{Tamayo Hernandez Fernando Yollotl }
    \section*{Modelo general de Programacion Lineal (notacion matricial)}
    \begin{itemize}
        \item Maximizar : \( z = cx \)
        \item Sujeto a : \( Ax \le b \) y \( x \ge 0 \)
            \begin{align*}
                x &= \left( x_1, x_2,\ldots,x_n \right)^T \\
                c &= \left( c_1 , c_2, \ldots , c_n \right) \\
                A &= (a_{ij})_{m\cdot n} \\
                b &= \left( b_1, b_2 , \ldots , b_m \right)^T
            .\end{align*}
    \item Donde \( c \) es aquello que representa el objetivo de la empresa.
    \item Maximizar :
        \[
        z = \sum_{j = 1}^{n} c_j x_j
    \] 
\item Sujeto a :
    \[
        \sum_{i = 1}^{m} \sum_{j = 1}^{n}    a_{ij} x_{j} \le  b_i \implies x_j \ge 0
    \] 
    \end{itemize}
    \section*{Problema de la WYNDOR GLASS } 
    El obejtivo de la WYNDOR GLASS es no importa lo que se fabrique , 
siempre se debe tener el maximo con lo minimo .\\
Siempre debemos ser muy especicficos con el nombre de nuestras variables.\\

\begin{align*}
    x &= \begin{pmatrix}
        x_1\\
        x_2
    \end{pmatrix}\\
        c &= \left( 3000 \ 5000 \right) \\
        b &= \begin{pmatrix} 
            4 \\
            12 \\
            8
        \end{pmatrix}\\
        A &=  
        \begin{pmatrix}
            1 & 0 \\
            0 & 2 \\
            3 & 2 
        \end{pmatrix}\\
        \text{Maximizar} &:z = ( 3000 \ 5000) \begin{pmatrix}
            x_1 \\
            x_2\\
        \end{pmatrix}
.\end{align*}
Sujeto a :
\[
        \begin{pmatrix}
            1 & 0 \\
            0 & 2 \\
            3 & 2 
        \end{pmatrix}
        \begin{pmatrix}
        x_1\\
        x_2
    \end{pmatrix} \le  \begin{pmatrix}
    4 \\
    12 \\
    18
\end{pmatrix}
.\] 
\begin{itemize}
    \item[] \( x_j  \) : número de lotes del producto \( j \) que se fabrican por semana.
        \item[] \( j = 1 \)puertas.
            \item[] \( j = 2 \) ventanas.
\end{itemize}
Maximizar : \( z = 3000\cdot x_1 + 5000 \cdot  x_2\)\\
Sujeto  a:
\begin{align*}
  & x_1 \le   4 \\
    & 2 \cdot x_2 \le  12 \\
    & 3 x_1 + 2 x_2 \le  18 \\
    & x_1 , x_2 \ge  0
.\end{align*}
\end{document}

