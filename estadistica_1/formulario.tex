\documentclass{article}
\usepackage[utf8]{inputenc}
\usepackage{fancyhdr}
\usepackage{mathtools}
\usepackage{unicode-math}
\usepackage{hyperref}
\usepackage[]{itemize}
\usepackage[]{enumerate}
\usepackage[
top  = 3.15cm,
bottom = 3.15cm,
left   = 3.15cm,
right  = 3.15cm]{geometry}

\pagestyle{fancy}
\fancyhf{}
\setlength{\headheight}{35pt}

\begin{document}
\rhead{Formulario }
\section*{Probabilidad condicional}
Probabilidad de \( A \) tal que \( B \):
\[
P(A|B) = \frac{P(A \cap B)}{P(B)}
.\] 
Probabilidad de intersección:
\[
    P(A \cap B) = P(A) + P(B) - P(A \cup B)
.\] 
\section*{Funciones de probabilidad}
\begin{enumerate}
    \item \textbf{Distribucion binomial negativa}
        \begin{itemize}
            \item Si llamamos \(X = “número de experimentos realizados hasta obtener el r-ésimo éxito”, diremos que la variable X sigue una distribución binomial negativa de parámetros r, p\).  
            \item entonces ponemos estoa ca
        \end{itemize}
\end{enumerate}
\end{document}
