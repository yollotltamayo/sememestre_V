\documentclass{article}
\usepackage[utf8]{inputenc}
\usepackage{fancyhdr}
\usepackage{mathtools}
\usepackage{unicode-math}
\usepackage{hyperref}
\usepackage[
top  = 3.15cm,
bottom = 3.15cm,
left   = 3.15cm,
right  = 3.15cm]{geometry}

\pagestyle{fancy}
\fancyhf{}
\setlength{\headheight}{35pt}

\begin{document}
\rhead{Tamayo Hernandez Fernando Yollotl \\ Tarea sobre logaritmos}
\section*{Ejercicio 1}
Cnociendo los logaritmos de 2, 3 ,5 y 7 .Calcula:
\begin{itemize}
    \item[(a)] \( \log 105 \)
        \begin{align*}
            \log 105 &= \log ( 7 \cdot 5 \cdot 3)\\
            &= \log 7 + \log 3 + \log 5
        .\end{align*}
    \item[(b)] \( \log 108 \)
        \begin{align*}
            \log 108 &= \log ( 3^3 \cdot 2^2)\\
            &= 3 \log 2 + 2 \log 2
        .\end{align*}
    \item[(c)] \( \log \sqrt[3]{72}  \)
        \begin{align*}
            \log \sqrt[3]{72} &= \frac{1}{3} \log ( 3^2 \cdot 2^3) \\
            &= \frac{1}{3}\left( 2 \log 3 + 3 \log 2  \right)  \\
            &= \frac{2}{3} \log 3 + \log 2
        .\end{align*}
    \item[(d)] \( \log 2.4\)
        \begin{align*}
            \log 2.4& = \log \frac{12}{5}\\
            &= \log 12 - \log 5 \\
            &= \log ( 2^2 \cdot 3) - \log 5 \\
            &= 3\log 2 + \log 3  - \log 5 \\
        .\end{align*}
\end{itemize}
\section*{Ejercicio 2}
Expresar las siguientes relaciones por un solo logaritmo.
\begin{itemize}
    \item[(e)] \( \log 2 - \log 3 + \log 5 \)
        \begin{align*}
            \log 2 - \log 3 + \log 5 &= \log 2 - \log 15 \\
            &= \log \frac{2}{15} 
        .\end{align*}
    \item[(f)] \( 3 \log 2 - 4 \log 3 \)
        \begin{align*}
            3 \log 2 - 4 \log 3 &= \log( 2^3) - \log(3 ^4) \\
                                &= \log \left( \frac{2^3}{3^4} \right)  
        .\end{align*}
    \item[(f)] \( \log 5 - 1 \)
        \begin{align*}
            \log 5 - 1 &= \log 5 - \log 10 \\
            &= \log \left( \frac{5}{10} \right)  \\
            &= \log \left( \frac{1}{2} \right)  \\
            &= -\log 2
        .\end{align*}
    \item[(h)] \( \frac{1}{3}\log 25 - \frac{1}{3}\log 64 + \frac{2 }{3}\log 27 \) 
        \begin{align*}
            \frac{1}{3}\log 25 - \frac{1}{3}\log 64 + \frac{2 }{3}\log 27 &=\log \sqrt[3]{25}  - \log \sqrt[3]{64}+\log \sqrt[3]{27^2}   \\
                                                                          &=\frac{ \log \sqrt[3]{25} \cdot \log \sqrt[3]{27^2} }{\sqrt[3]{64}}
        \end{align*}
    \item[(i)] \( 2 \log 3 + 4 \log_2 - 3 \)
\end{itemize}
\end{document}
